\documentclass[a4paper, 12pts]{article}

\usepackage[top=3.5cm, bottom=3.5cm, left=3cm, right=3cm]{geometry}

\usepackage[T1]{fontenc}
\usepackage[utf8]{inputenc}
\usepackage[francais]{babel}
\usepackage{textcomp}
\usepackage{listings}
\usepackage{authblk} %author tools

%\usepackage{hyperref} %pour les liens internet

%\usepackage{graphicx} %pour les images
\title{Fil Rouge}
\author{Sebastien DI GIOVANNI}
\author{Bruno GODEFROY}
\author{Edern HAUMONT}
\author{François ROBION}
\author{Nicolas SIX}
\affil{}
\date{\today}

%-----------------------------------------------------------------------------------------


\begin{document}

%\begin{titlepage}

\maketitle

%\end{titlepage}

%----------------------------------------------Title end

\section{Introduction}
\paragraph{}
	Le but de ce projet est d'explorer un domaine de l'informatique novateur, très présent dans l'imaginaire collectif et qui pourtant nous paraît encore très étranger du point de vue de son fonctionnement : l'intelligence artificielle via des réseaux de neurones. Les intelligences artificielles se contentant de tester plusieurs/toutes les possibilités afin d'en choisir la meilleure suivant certains critères ayant déjà été explorés par plusieurs membres du groupe, nous aimerions maintenant essayer d'aller plus loin et d'explorer l'art d'apprendre à apprendre. La finalité de ce projet ne consiste pas à essayer de réinventer des techniques existantes à partir de zéro, mais bien d'emmagasiner des connaissances dans un domaine en plein essor et de se confronter aux problèmes liés à de tels systèmes, via l'implémentation d'un modèle simple.

\section{Prétexte}
\paragraph{}
	Pour mettre en pratique nous recherches, nous essaierons de mettre en oeuvre un modèle simple mais aux possibilités immenses, issus du monde réel. C'est ainsi que nous avons choisi ce que nous appellerons la savane, un environnement dans lequel plusieurs espèces d'entités évoluent et coopèrent. Certaines entités, “carnivores”, auront tendance à y développer des techniques de chasse alors que d'autres, “herbivores”, de troupeau ou d'esquive. Chaque entité développant sa propre intelligence avec la possibilité de la transmettre partiellement à sa descendance.

\section{Comment}
\paragraph{}
	Dans cette optique nous voulons donc développer un réseau de neurones capable d'évoluer par lui-même, d'apprendre de son environnement et ainsi de s'adapter. Étant conscient du travail qu'un tel projet représente, nous voulons d'abord tenter de mettre ce principe en place sur une entité qui apprendrait d'abord seule ou grâce à une autre entité contrôlée par un humain. Seulement ensuite, nous essayerons de confronter plusieurs espèces différentes, avec l'espoir de pouvoir observer, à terme, une évolution des espèces d'un point de vue intellectuel, mais aussi biologique par la séparation d'espèces en sous-espèces...

\end{document}
